\chapter{Przykład kompilacji źródeł odtwarzacza}
\label{cha:dodatekB}

Na listingu \ref{lst:cmd} przedstawiono komendę z której korzysta program Microsoft Visual Studio 2010 Professional w celu kompilacji źródeł odtwarzacza multimedialnego.

\begin{lstlisting}[caption=Kompilacja źródeł odtwarzacza przez program MVS 2010., label=lst:cmd]
...> cmake /I"C:\boost_1_55_0" /Zi /nologo /W3 /WX- /O2 /Oi /Oy- /GL
/D "WIN32" 
/D "NDEBUG" 
/D "_CONSOLE" 
/D "_UNICODE" 
/D "UNICODE" 
/Gm- /EHsc /GS /Gy /fp:precise /Zc:wchar_t /Zc:forScope 
/Fp"C:\...\libdash\intermediate\clientdash\ReleaseWin32\clientdash.pch" 
/Fa"C:\...\libdash\intermediate\clientdash\ReleaseWin32\" 
/Fo"C:\...\libdash\intermediate\clientdash\ReleaseWin32\" 
/Fd"C:\...\libdash\intermediate\clientdash\ReleaseWin32\vc100.pdb" 
/Gd /analyze- /errorReport:queue 
\end{lstlisting}
